\documentclass{beamer}
%
% Choose how your presentation looks.
%
% For more themes, color themes and font themes, see:
% http://deic.uab.es/~iblanes/beamer_gallery/index_by_theme.html
%
\mode<presentation>
{
  \usetheme{default}      % or try Darmstadt, Madrid, Warsaw, ...
  \usecolortheme{default} % or try albatross, beaver, crane, ...
  \usefonttheme{default}  % or try serif, structurebold, ...
  \setbeamertemplate{navigation symbols}{}
  \setbeamertemplate{caption}[numbered]
} 

\usepackage[english]{babel}
\usepackage[utf8]{inputenc}
\usepackage[T1]{fontenc}

\usepackage{amsmath, amssymb, physics, tikz, tikz-feynman, quantikz}
\usepackage{graphicx, subcaption}

\title{Tensor Networks : Session 1}
\author{Hanse Kim}

\begin{document}

\begin{frame}
  \titlepage
\end{frame}

%\begin{frame}{What is a Tensor?}
%	\begin{itemize}
%		\item Tensor in physics : something that \textit{transforms like a tensor}
%			\begin{itemize}
%			\item Arises from a more mathematical definition
%			\end{itemize}
%
%			\item Vector space ${ V }$ : define as rank (1,0)-tensor
%				\begin{itemize}
%			\item Transformations ${ GL(V) }$ are induced by coordinate transformations
%			\end{itemize}
%			\item Linear operators ${ V^{*} }$ : space of all functionals ${ V\to \mathbb{R}  }$, define as rank (0,1)-tensor
%			\item Rank ${ (p,q) }$-tensors built from this structure (via \textbf{tensor products})
%				\begin{itemize}
%					\item Transformation properties are also induced
%				\item ex. Metric tensor ${ g_{\mu \nu } }$ : (0,2)-tensor
%				\end{itemize}
%	\end{itemize}
%\end{frame}

\begin{frame}{What is a Tensor?}
\begin{itemize}
	\item Tensor in physics : something that \textit{transforms like a tensor}
		\begin{itemize}
		\item Arises from a more mathematical definition
		\item We are interested in a more simple definition
		\end{itemize}

		\item Vector space ${ V }$ : define as rank 1 tensor
		\item \textbf{Tensor product} : rank ${ n }$ ${ \otimes  }$ rank ${ m }$ ${ = }$ rank ${ (m+n) }$ tensor
			\begin{itemize}
			\item Components : product; ${ (A \otimes B)_{ij}=A_{i}B_{j} }$
			\item cf. Cartesian product; do not confuse rank and dimensionality!
			\end{itemize}
\end{itemize}
\end{frame}

\begin{frame}{Diagrammatic Notation}
	\begin{itemize}
		\item Indices are written as legs
		\begin{itemize}
		\item Number of legs = rank
		\item Dimensionality of each index is not explicitly shown
		\item Can express components of tensor by writing indices
		\end{itemize}
	\item Product between tensors : write tensors side-by-side
	\item Contract over two indices (of same dimensionality) by connecting indices
	\end{itemize}
\end{frame}

\begin{frame}{What are Tensor Networks?}
	\begin{itemize}
	\item Consider many-body quantum system; ex. 1D lattice of $N$ spin-$1/2$ particles
		\begin{itemize}
			\item Each site has $p=2$ \textbf{degrees of freedom}
		\end{itemize}
	\item General state vector : specify components for all possible configurations
	\item State vector is given by specifying components of a \textbf{tensor}
	\[
		\ket{\Psi } =\sum_{\substack{s_{j}=\{+,-\}\\j=1\dots N}}^{} T_{s_{1}\cdots s_{N}}\ket{s_{1}}\otimes \cdots \ket{s_{N}}
	\]
	\begin{itemize}
	\item Total ${ p^{N} }$ degrees of freedom; exponential in ${ N }$
	\end{itemize}
	\end{itemize}	
\end{frame}

\begin{frame}{What are Tensor Networks?}
	\begin{itemize}
		\item Limited class of state vectors : specify \textbf{state vectors} at each site; take \textbf{tensor product} of all vectors
			\begin{align}
		\ket{\Psi } &= \left( \sum_{s_{1}=\{+,-\}}^{} A_{s_{1}}\ket{s_{1}} \right) \otimes \cdots \left( \sum_{s_{p}=\{+,-\}}^{} A_{s_{p}}\ket{s_{p}} \right) \\  
					&= \sum_{\substack{s_{j}=\{+,-\}\\j=1\dots N}}^{} \tilde{T}_{s_{1}\cdots s_{N}}\ket{s_{1}}\otimes \cdots \ket{s_{N}} \\
	\end{align}
	\item State vector is also a \textbf{tensor}; not all components are 'independent'!
		\begin{itemize}
		\item Total ${ pN }$ degrees of freedom; polynomial in N
		\item Does not generate the entire Hilbert space
		\end{itemize}
	\end{itemize}
\end{frame}

\begin{frame}{What are Tensor Networks?}
	\begin{itemize}
		\item Limited class of state vectors : specify \textbf{tensors} at each site; \textbf{contract} indices except one
		\item Most general kind of tensor network
			\begin{itemize}
				\item Dimensionality of contracted indices : \textbf{bond dimension} ${ D }$
				\item Typically degrees of freedom polynomial in ${ N }$
			\end{itemize}
		\item Examples
			\begin{itemize}
				\item \textbf{Matrix Product State} : 1D array of tensors
				\item \textbf{Projected Entangled Pair States} : 2D array of tensors
			\end{itemize}
		
	\end{itemize}
\end{frame}

\begin{frame}{Why Tensor Networks?}
	\begin{itemize}
	\item Using tensor networks : considering only wave functions that can be written in this form
	\item Why limit ourselves to tensor network states?
	\end{itemize}
\end{frame}

\begin{frame}{Why Tensor Networks?}
	\begin{itemize}
		\item \textbf{Natural representation of entanglement}
			\begin{itemize}
			\item TNs are natural representations for many-body wave functions that are visually intuitive
			\item TN structure means components of wavefunction are not 'independent'
			\item This dependence characterises entanglement present in states
			\end{itemize}
	\end{itemize}
\end{frame}

\begin{frame}{Why Tensor Networks?}
	\begin{itemize}
		\item \textbf{Hilbert space is 'too large'}
			\begin{itemize}
			\item (Fortunately,) not all states in the Hilbert space are relevant
				\begin{itemize}
			\item Why? Locality (states of 'neighbouring' particles are not completely independent)
			\item In particular, low energy states follows the entropy area law
			\end{itemize}
		\item Number of states typically are polynomial in ${ N }$
			\end{itemize}
	\end{itemize}
\end{frame}

\begin{frame}{Some General Notes}
	\begin{itemize}
	\item Complexity of contracting a TN is order dependent
		\begin{itemize}
		\item Only relevant if bond dimensions are nonuniform
		\item Expectation values of TNs is a contraction of two TNs (+ some operators)
		\end{itemize}
	\item Given a chosen form of TN, there is still choice of bond dimension ${ D }$ that can be made
		\begin{itemize}
		\item For ${ D }$ sufficiently large, ex. MPS can eventually cover the whole Hilbert space
		\item For ${ D =1}$, no entanglement is present; used in mean field theory
		\end{itemize}
	\end{itemize}
\end{frame}

\begin{frame}{Physical Properties : MPS}
	\begin{itemize}
		\item \textbf{MPS} : 1D array of tensors
		\begin{itemize}
		\item Two examples; open BC, periodic BC
		\item One tensor per site in many-body system; open indices represents physical degrees of freedom
		\end{itemize}
	\item MPS are dense : any Hilbert state can be represented by increasing ${ D }$
	\item Area law : entanglement entropy is constant wrt. ${ L }$
	\end{itemize}
\end{frame}


\end{document}

