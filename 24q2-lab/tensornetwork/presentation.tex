\documentclass{beamer}
%
% Choose how your presentation looks.
%
% For more themes, color themes and font themes, see:
% http://deic.uab.es/~iblanes/beamer_gallery/index_by_theme.html
%
\mode<presentation>
{
  \usetheme{default}      % or try Darmstadt, Madrid, Warsaw, ...
  \usecolortheme{default} % or try albatross, beaver, crane, ...
  \usefonttheme{default}  % or try serif, structurebold, ...
  \setbeamertemplate{navigation symbols}{}
  \setbeamertemplate{caption}[numbered]
} 

\usepackage[english]{babel}
\usepackage[utf8]{inputenc}
\usepackage[T1]{fontenc}

\usepackage{amsmath, amssymb, physics, tikz, tikz-feynman}
\usepackage{graphicx}

\title{Tensor Networks}
\author{Hanse Kim}

\begin{document}

\begin{frame}
  \titlepage
\end{frame}

\begin{frame}{Tensor Networks}
	\begin{itemize}
	\item Framework of decomposition representation for many-body wavefunctions
	\item Diagrammatic language for quantum physics
	\item Explicit and accessible representation of entanglement between constituents in many-body
	\item Relevant degrees of freedom for low-energy states corresponds to TN states
	\end{itemize}	
\end{frame}

\begin{frame}{Tensor Network}
	\begin{itemize}
	\item \textbf{Tensor network} : product of tensors with contracted indices
	\item \textbf{Diagrammatic notation}
	\\\phantom{\includegraphics[width=0.5\textwidth]{example-image-a}}
	% Diagrams
	%\item \textbf{Property} : computational cost of contraction is order dependent
	\end{itemize}
\end{frame}

\begin{frame}{Wavefunction Decomposition}
	\begin{itemize}
	\item $N$ particles with $p$ states : tensor product basis
		\[
			\ket{\Psi }=\sum_{i_{j}=1\dots p}^{} C_{i_{1}\dots i_{N}}\ket{i_{1}}\otimes\cdots\ket{i_{N}}
		.\]
		\begin{itemize}
		\item $N$ rank tensor; $p^{N}$ (exponential) DoF
		\item Not all coefficients are independent in presence of entanglement
	\\\phantom{\includegraphics[width=0.3\textwidth]{example-image-a}}
			% Simple example : two spin 1/2 particles entangled into singlet state
		\end{itemize}	
	\end{itemize}
\end{frame}
\begin{frame}{Wavefunction Decomposition}
	\begin{itemize}
	\item Decompose tensor into tensors of smaller rank
		% Example : rank N tensor
		% Product of N vectors (naive case of both below)
		% MPS with periodic bc
		% PEPS with open bc
	\\\phantom{\includegraphics[width=0.5\textwidth]{example-image-a}}
		\begin{itemize}
		\item $pN$ (polynomial) DoF
		\item Additional DoF contribution due to dimensionality of contracted indices; \textbf{bond indices}
		\end{itemize}
	\end{itemize}
\end{frame}

\begin{frame}{Wavefunction Decomposition}
	\begin{itemize}
	\item Bond indices represents entanglement
	\item ex. Entanglement entropy between boundary and inner tensors of PEPS system
		% Diagram
	\\\phantom{\includegraphics[width=0.3\textwidth]{example-image-a}}
		\[
		S(L)=-\text{tr}(\rho_{\text{in}}\log{\rho_{\text{in}}})\leq 4L \log{D}
	.\]
	\begin{itemize}
	\item For trivial TNs ($D=1$), $S(L)=0$; no entanglement present
	\item $D>1$ : area-law for entanglement entropy
		%form of $S(L)$ dependent on connection geometry; constant dep on D
	\end{itemize}
	\end{itemize}
\end{frame}

\begin{frame}{Matrix Product States}
	\begin{itemize}
		\item \textbf{MPS} : One-dimensional array of tensors
	\\\phantom{\includegraphics[width=0.3\textwidth]{example-image-a}}
		% Diagrams for open and periodic
		\begin{itemize}
		\item Two examples; open BC, periodic BC
		\item One tensor per site in many-body system; open indices : physical DoF
		\end{itemize}
	\item Properties
		\begin{itemize}
		\item MPS are dense : any Hilbert state represented by increasing $D$; low energy states in 1D efficiently approximated with $D$ polynomial in $N$
			% for gapped local Hamiltonians
		\item Area-law for entanglement entropy; ground states for large $N$
		\end{itemize}
	\end{itemize}
\end{frame}
\begin{frame}{Matrix Product States}
	\begin{itemize}
		\item Properties
		\begin{itemize}
		\item Finitely correlated; $\ev{O O^{\prime}}-\ev{O}\ev{O^{\prime}}\sim f(r)\exp -r/\xi$  
		\end{itemize}
	\end{itemize}
	\\\phantom{\includegraphics[width=0.5\textwidth]{example-image-a}}
\end{frame}

\begin{frame}{Matrix Product States}
	\begin{itemize}
		\item Examples
	\\\phantom{\includegraphics[width=0.5\textwidth]{example-image-a}}
		% Diagrams for open and periodic
	\begin{enumerate}
		\item GHZ State : $\ket{\Psi}=\frac{1}{\sqrt{2}}(\ket{0}^{\otimes \infty}+\ket{1}^{\otimes \infty})$
		\item 1D cluster state : simultaneous eigenket of $K^{i}=S^{z}_{i-1}S^{x}_iS^z_{i+1}$; $\ket{\Psi }=\prod_i \frac{1+K^{i}}{2}\ket{0}^{\otimes \infty}$
		\item Ground state of 1D AKLT : $H=\sum_{\ev{i,j}}^{}\boldsymbol{S}_i\cdot\boldsymbol{S}_j+\frac{1}{3}(\boldsymbol{S}_i\cdot\boldsymbol{S}_j)^2 $
		\end{enumerate}
	\end{itemize}
\end{frame}

%\begin{frame}{Projected Entangled Pair States}
%	\begin{itemize}
%		\item \textbf{PEPS} : Two-dimensional array of tensors
%	\\\phantom{\includegraphics[width=0.3\textwidth]{example-image-a}}
		% Diagrams for open and periodic
%		\begin{itemize}
%		\item 2D lattices, tensor renormalisation group etc.
%		\end{itemize}
	%\item Properties
	%	\begin{itemize}
	%	\item 
	%	\end{itemize}
%	\end{itemize}
%\end{frame}

\begin{frame}{Density-Matrix Renormalisation Group}
	\begin{itemize}
	\item 1D spin $\frac{1}{2}$ lattice; iterative method of increasing sites (sweep)
	\begin{itemize}
		\item Split lattice into two blocks $A,B$ + two intermediate sites; \textbf{superblock} $A\cdot\cdot B$ % each site in block is dim d; intermediate site is dim D(3)
		\item Diagonalise superblock Hamiltonian; find ground state $H=-J\sum_{\ev{i,j}}^{} \boldsymbol{S}_{i}\cdot\boldsymbol{S}_{j} -h\sum_{i}^{} S^z_{i}$\\
			$\ket{\Psi }_{G}=\sum_{ij}^{} \psi_{ij}\ket{i}_{A\cdot}\ket{j}_{\cdot B}$; basis of block $A\cdot$ $\ket{i}_{A\cdot}$
		\item Diagonalise reduced density operator for $A\cdot$ given ground state \\
			$\rho_{A\cdot}=\text{Tr}_{\cdot B}\ket{\Psi}_G\bra{\Psi}_G$; $(\rho_{A\cdot})_{ij}=\sum_{k}\psi_{ik}\psi^\star_{jk}$\\
			Effective basis of new block $A^{\prime}$ $\ket{i^{\prime}}_{A^{\prime}}$
		\item Truncate for largest eigenvalues to reduce dimensionality
		\item Update operators to new basis\\
			$\ev{i^{\prime}| O|j^{\prime}}=\sum_{ij} \ev{i^{\prime}|i}\ev{i|O|j}\ev{j|j^{\prime}}$
	\end{itemize}
\item Ansatz for superblock Hamltionian ground state : MPS
\end{itemize}
\end{frame}

\begin{frame}{MPS and Machine Learning}
\begin{itemize}
\item Feature vector constructed from tensor products; weights constructed from tensor networks (ex. MPS)
	\[
	\mathbf{v}(x)= \begin{bmatrix}
	\phi_1(x_{1}) \\ \phi_2(x_1)\\
	\end{bmatrix}\otimes \begin{bmatrix}
	\phi_1(x_{2}) \\\phi_2(x_2)\\
	\end{bmatrix}\otimes\cdots\otimes\begin{bmatrix}
	\phi_1(x_{N}) \\ \phi_2(x_N)\\
	\end{bmatrix}
	.\]
	\[
		f(x)=W\cdot \mathbf{v}(x)=\sum_{s_{i}=\{1,2\}}W_{s_{1}\cdots s_{N}}\phi_{s_{1}}(x_{1})\cdots\phi_{s_{N}}(x_{N})
	.\]
	
	
\end{itemize}	
%\\\phantom{\includegraphics[width=0.5\textwidth]{example-image-a}}
\end{frame}

\begin{frame}{MPS and Machine Learning}
\begin{itemize}
\item ex. Unsupervised tree tensor network training (data compression) + supervised training of top layer (classification)
\item $$
\end{itemize}	
\\\phantom{\includegraphics[width=0.5\textwidth]{example-image-a}}
\end{frame}
\end{document}
